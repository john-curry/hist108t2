\documentclass[a4paper,12pt]{article}
\usepackage{times}
\usepackage{setspace}
\usepackage[backend=biber,url=false]{biblatex-chicago}
\usepackage[margin=1.0in]{geometry}
\addbibresource{hist108t2.bib}
\doublespacing
\nonstopmode
\title{}
\author{John Curry\\
        C0409094\\
        History 108\\
        Camosun College\\
        }
\date{\today}

\begin{document}
\maketitle
%when asked to find a way to make jet engines 50\% more efficient, NASA used private companies GE and Pratt \& Whitney

When humanity started to reach for the stars, all eyes were on the skies. The fact that there were Communist satellites in space was suddenly a talking point for the world. Everyone started to ask, "What does this mean?", and indeed the question hung in the air while people debated what this meant for the future of the human race. In the end it came down to the United States of America making sweeping proclamations about about the threat from the sky and how they were going to win the space race. Although the reasons for starting the US' mad dash for the moon was awash with the need to broker good propaganda against the Soviets, it started a international conversation about what humanities purpose is in space, how we are going to get there, and what to do when we are there. To the US the conversation ended at "We are there to fight the Russians, we are going to get there with money and science, and we are going to win". 

The move humanity has taken from the ground to the skies has been fraught with political implications, international maneuvering and public outcry, but it has push humanity to the next great frontier. The development of space exploration technology will drive scientific advancement, nurture international cooperation, boost international and national economies as well as help solve some of humanities greatest problems. 

The story of space is a story of firsts. History is read to us from the beginning and those who are to be written in our history books want to be there right at the beginning. When the USSR got the first man made object in space, there was considerable push to make it so the Americans were to be the first in the next coming set of firsts. Unfortunately there were a few that they missed. The Soviets got the first animal in space, first man in space and first woman in space \autocite{west2001}. The Americans eventually got their man on the moon, and their place in the history books. In the US it is well known that the launch of Sputnik 1, the USSR' first man made satellite, was the hot topic of the times, but in the Soviets eyes in was only a minor achievement \autocite{west2001}. The Soviets were set on getting to the moon, and Sputnik 1 was only a small step in that direction. In the background though, the fight for space was a fight for technical superiority which meant advancements in th areas of flight propulsion, medicine in low gravity environment, and a myriad amount of other scientific disciplines. 

When the USSR sent their dog Laika into space, they used it as a testing bed for sending people into space. They used it to test electrocardiogram, blood pressure, respiration rate and motor activity \autocite{west2001}. There would be the first steps to making sure it was safe for humans to be in space. To understand the importance of these test one needs to understand the implications of whether or not people could live in space. With the growing word population, if conditions were right, population control would not be a problem as people could start colonizing space. Although indeed it would be safe to send a living being into space, the USSR was only doing this to get a man on the moon \autocite{west2001}. The Soviets sent up the first man into space, and later the first manned space station.

Nowadays, space exploration and research is being driven by international cooperation instead of being a theater for war. The International Space Station (ISS) is an international collaboration that started in 1993 that has modules built by the US, uses rockets built by Russia, has seen tourists from South Africa, been resupplied by shuttles from Japan and also has modules built by the European Space Agency (ESA) \autocite{funk2014}. It seems everyone is in space these days, not just the old Soviet-American conflict of the old days. Even the Chinese are in space, having sent numerous probes into space as well as having plans for spaceports, manned missions and space stations \autocite{jolly2013}. Indeed the international community is both building their own space exploration technology, but also working together in a manner unprecedented. International cooperation is a good thing. When people are working together on a common goal, it puts them on common ground and if there is conflict they have to think about what they have to lose if their partnership breaks down. 

Development of these technologies also brings about a massive amount of money into private industry. Building projects on these scales drives research into areas that may not have been researched otherwise or may drive research that would otherwise be underfunded. 


\newpage
\printbibliography
\end{document}
