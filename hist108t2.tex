\documentclass[a4paper,12pt]{article}
\usepackage{times}
\usepackage{setspace}
\usepackage[backend=biber,url=false]{biblatex-chicago}
\usepackage[margin=1.0in]{geometry}
\addbibresource{hist108t2.bib}
\doublespacing
\nonstopmode
\title{}
\author{John Curry\\
        C0409094\\
        History 108\\
        Camosun College\\
        }
\date{\today}

\begin{document}
\maketitle
\nocite{*}
\section*{Thesis}
given us unprecedented information about the universe we live in\\
given a boost to the private and public aeronautics industry\\
%when asked to find a way to make jet engines 50\% more efficient, NASA used private companies GE and Pratt \& Whitney
scientific advancement in a wide range of fields \\
international cooperation\autocite{woodsbrian}\\

When humanity started to reach for the stars, all eyes were on the skies. The fact that there were Communist satellites in space was suddenly a talking point for the world. Everyone started to ask, "What does this mean?", and indeed the question hung in the air while people debated what this meant for the future of the human race. In the end it came down to the United States of America making sweeping proclamations about about the threat from the sky and how they were going to win the space race. Although the reasons for starting the US' mad dash for the moon was awash with the need to broker good propaganda against the Soviets, it started a international conversation about what humanities purpose is in space, how we are going to get there, and what to do when we are there. To the US the conversation ended at "We are there to fight the Russians, we are going to get there with money and science, and we are going to win". 

The move humanity has taken from the ground to the skies has been fraught with political implications, international maneuvering and public outcry, but it has push humanity to the next great frontier. The development of space exploration technology will drive scientific advancement, nurture international cooperation, boost international and national economies as well as help solve some of humanities greatest problems. 
\newpage
\printbibliography
\end{document}
